\chapter{研究现状}
\label{cha:china}
本章对跨语言对齐与知识库构建涉及到的相关技术进行了研究与探讨,主要对schema对齐、跨语言实体对齐跨语言属性对齐以及一些知名知识库的构建流程与技术上给出了详细的介绍,在这些研究的启发下,结合实际问题与切实数据情况,提出属性对齐方法与跨语言知识库的构建流程。

\section{本章引论}
自从Time Berner-Lee提出语义万维网的概念并规划出发展前景,语义网逐渐受到研究领域的关注,开始快速发展。通过语义万维网的直接目标:即让计算机理解语义文本,加强自动交流,可以预见语义网的建立要结合人工智能与Web技术。以自然语言处理、信息检索、机器学习为首的人工智能负责对纯文本进行分析、清理、语义理解、知识融合等工作,而Web技术则主导解析HTML文本,以及为知识关联与利用提供语义化环境等。从当前技术的发展趋势来看,作为人工智能与Web相结合的产物,语义网的出现,也是万维网发展的必然结果。

知识图谱是语义网发展过程中的产物。这一概念最初由谷歌提出,是谷歌对其建立的知识库产品的称呼,其作用为理解用户搜索内容,从更深、更广的角度提供一个全面的搜索结果。现在这一名称被广泛用来指代大规模知识库。知识图谱涵盖大量实体,同时以RDF三元组,即主语-谓语-宾语的形式存储着与实体与其他实体、概念、属性之间的关系。一般来说,一个实体属于一个或多个概念,还含有多个facts,表征该实体的特性。与精准度高的本体不同的是,知识图谱规模庞大,一般从多个数据源中精炼而成,具有一定的容错性,如何自动融合多源知识,如何提取正确关系,一直是知识图谱不可避免的阻碍,也是该领域广受关注的研究点。发布于开放链接(LOD)项目中的大规模开放数据集,都可称之为知识图谱。本章将对现今知识库现状进行讨论,并列举比较典型或相关的知识库,探讨其优缺点,xxx(详见\ref{sec:knowledgebase-research}),并从实体的角度,介绍其目前常见问题与应用(详见\ref{sec:entity-research})。

针对从海量数据分析中构建知识图谱这一问题,YAGO作者在文献\cite{suchanek2014knowledge} 中提及,其中一项有着重要意义的工作为获取关系事实。关系事实表征着实体的属性,以及实体与其他实体的关系,比如“清华大学-现任校长-邱勇”,即为一个关系事实。规模较小的领域本体,因实体类型限制在一定领域,其涉及的属性也屈指可数,一般可以通过人工直接定义\cite{boyce2007developing}或人工抽取\cite{王巍巍2016双语影视知识图谱的构建研究}来获得。然而以百科等大型数据集为来源的知识图谱,通常从页面信息框中提取关系事实与属性,数量在万级以上,杂质较多,准确度不能保证。基于此,对百科信息框属性的研究也数不胜数。\ref{sec:property-research} 中将会从跨语言属性对齐、属性模板、xxx方面来阐述近些年对百科信息框属性的研究,这些工作与知识库构建息息相关,对我们的属性相关任务有着很大的指导意义。

\section{百科信息框属性相关研究}
\label{sec:property-research}

\subsection{跨语言schema对齐}

\subsection{跨语言属性对齐}


\section{现有知识库构建和相关技术}
\label{sec:knowledgebase-research}

随着语义网概念的普及与技术的发展,各种领域、不同规模的知识库层出不穷。发布在链接数据(LOD)上的知识库林林总总,既有限定领域的LinkedMDB\cite{erxleben2014introducing}、GeoNames\cite{wick2011geonames},也有着著名的大型多语言知识库如Dbpedia、Freebase、Wikidata等。这些知识库为实现统一与共享全世界知识的远景做着不懈的努力。
除此之外,一些知名公司也在做着相关产品,比如知识图谱的领头羊,谷歌的Google Knowledge Graph\cite{singhal2012introducing},微软研究院的EntityCube\cite{nie2012statistical}和Probase\cite{wu2012probase},以及IBM的Watson项目\cite{ferrucci2012introduction}


\subsection{中文知识库}
国内存在一些商业化的中文知识库,如搜狗“知立方”和百度“知心”,它们都是以人性化搜索、个性化推荐为出发点所建立的知识图谱产品。此外,也有不少企业利用知知识图谱的语义关系,建立与查找基于用户的知识图谱,从而可以实现在社交网络上推荐好友,在信贷领域识别借贷人等功能。

\subsection{多语言知识库}

\subsection{基于海量网页数据的知识库}


\section{实体相关技术}
\label{sec:entity-research}

\subsection{实体识别与实体消歧}
\label{sec:}

\subsection{实体链接}

\section{本文研究目标和思路}

\subsection{基于百科的跨语言属性对齐}
\label{sec:}

\subsection{大规模中英文跨语言知识库XLore的构建}
\label{sec:}

\subsection{XLore系统与应用接口构建}
\label{sec:}

\section{本章小结}


