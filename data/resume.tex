\begin{resume}

  \resumeitem{个人简历}

  1991 年 09 月 27 日出生于河北省廊坊市大厂回族自治县。

  2009 年 9 月考入北京邮电大学软件学院攻读软件工程专业,于 2013 年 7 月本科毕业并获得工学学士学位。 

  2013 年 9 月作为免推研究生,进入清华大学计算机科学与技术系,攻读计算机技术专业硕士至今,在知识工程实验室工作。

  \researchitem{发表的学术论文} % 发表的和录用的合在一起

  % 1. 已经刊载的学术论文(本人是第一作者,或者导师为第一作者本人是第二作者)
  \begin{publications}
    \item Li M., Shi Y., Wang Z. and Liu Y.. Building a Large-Scale Cross-Lingual Knowledge Base from Heterogeneous Online Wikis. Natural Language Processing and Chinese Computing. Springer International Publishing, 2015, 413-420.
    \item Wang Z., Li J., Li S., Li M., Tang J., Zhang K., and Zhang K.. Cross-lingual Knowledge Validation Based Taxonomy Derivation from Heterogeneous Online Wikis. AAAI'14. pp. 180-186.
    \item Wang Z., Li J., Wang Z., Li S., Li M. et al. Xlore: A large-scale english-chinese bilingual knowledge graph. In Proceedings of the 2013th International Conference on Posters \& Demonstrations Track-Volume 1035 (pp. 121-124). CEUR-WS. org.
  \end{publications}

  % 2. 尚未刊载,但已经接到正式录用函的学术论文(本人为第一作者,或者
  %    导师为第一作者本人是第二作者)。
  %\begin{publications}[before=\publicationskip,after=\publicationskip]
  %  \item Yang Y, Ren T L, Zhu Y P, et al. PMUTs for handwriting recognition. In
  %    press. (已被 Integrated Ferroelectrics 录用. SCI 源刊.)
  %\end{publications}

  % 3. 其他学术论文。可列出除上述两种情况以外的其他学术论文,但必须是
  %    已经刊载或者收到正式录用函的论文。
  %\begin{publications}
  %  \item Wu X M, Yang Y, Cai J, et al. Measurements of ferroelectric MEMS
  %    microphones. Integrated Ferroelectrics, 2005, 69:417-429. (SCI 收录, 检索号
  %    :896KM)
  %  \item 贾泽, 杨轶, 陈兢, 等. 用于压电和电容微麦克风的体硅腐蚀相关研究. 压电与声
  %    光, 2006, 28(1):117-119. (EI 收录, 检索号:06129773469)
  %  \item 伍晓明, 杨轶, 张宁欣, 等. 基于MEMS技术的集成铁电硅微麦克风. 中国集成电路,
  %    2003, 53:59-61.
  %\end{publications}

  \researchitem{研究成果} % 有就写,没有就删除
  \begin{achievements}
    \item 李涓子,王志刚,李双婕,李明洋,唐杰. 跨语言本体构建方法及装置: 中国, CN103336852A. 2013.10.02
  \end{achievements}

\end{resume}
