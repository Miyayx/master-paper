\chapter{基于维基百科的概念属性生成}
\label{cha:concept-property}
本章从维基百科中生成概念属性,借由信息框模板抽象概念,并生成属性集合。通过详细解析维基百科信息框属性的特点,解决了模板定义多样化、显示信息与模板信息不一致造成的获取困难。

\section{本章引论}

Schema作为知识库骨架,在知识库构建过程中占有重要地位。概念和属性是Schema中的重要元素,概念集合一类相似实例,属性表征实例特征。特定概念下的实例由一组共同的属性描述,获取这些属性集合,有利于我们形成准确的Schema,进而很好的描述现实事物。
维基百科中的信息框模板为基于维基的知识库的概念属性生成提供了丰富的信息。维基百科鼓励编辑者使用模板对词条及信息框进行组织与编辑。模板是百科针对不同主题的词条内容所列出的标准结构框架,是长期积累形成的标准写作规范。模板使词条的结构变得有规律可循,也可以有效避免关键信息的缺失。信息框的编辑也可模板化,信息框模板包含了描述一类实例的常用属性。图\ref{fig:template-infobox-film}为\textit{Template:电影信息框(Template:Infobox Film)}\footnote{\url{https://zh.wikipedia.org/wiki/Template:电影信息框}},其中含有电影的诸多属性,如\textit{演员,导演,时长}等。模板在一定程度上描述了一个领域的信息,充分利用其中信息,对概念-属性关系的研究有很大意义。

另一方面,属性通常带有一定的领域信息,即不同概念下的属性意义可能不一样。如标签为\textit{产地}的属性,在电影领域中表示\textit{制作国家},而在植物领域中则表示\textit{生长地点}。基于维基百科构建的知识库通常通过信息框内容扩展属性,但并未对属性进行明显的领域区分,相同标签的属性往往一视同仁。以概念领域划分属性,可以有效避免属性歧义,此时属性的意义也依赖于概念,同一个概念下的属性含义唯一。

\begin{figure}[ht]
  \centering
  \fbox{\includegraphics[width=0.8\columnwidth]{template-infobox-film}}
  \caption{电影信息框模板}
  \label{fig:template-infobox-film}
\end{figure}

概念属性生成工作是其他属性研究的基础,其结果的数量与质量在很大程度上影响着其他研究的准确率,过多杂质的引入会增加后续工作的难度。

然而百科数据的杂乱性,使得概念属性生成工作注定繁琐,本章通过对维基百科数据文件中的模板信息进行大量观察与分析,总结出模板与属性的定义规律,基于模式匹配提取概念属性,其结果覆盖了90\%以上的维基词条,为今后对信息框属性的进一步分析打下了良好的基础。

本章主要对信息框属性进行研究,后文中提到的{\heiti 模板},如无特别说明,统一指代{\heiti 信息框模板}。

\section{维基百科信息框模板与属性分析}
\label{sec:template-analysis}

维基百科自2001年创立,经过15年左右的发展,不仅储备了海量数据,在编辑规则上也越发规范。但因为词条繁杂、参与人数众多,还是无法保证数据统一,使得从中抽取概念属性的工作困难重重。具体来说,主要包括模板获取过程中的模板结构差异问题,以及属性获取过程中的显示标签与模板标签不一致问题。

{\heiti 模板获取问题:}
模板在百科中是如何定义的呢?模板在维基百科中也设计成一个词条,在维基数据文件中,含有对模板词条的具体定义及其使用说明。模板词条标签以\textit{Template:}开头,大部分的信息框模板标签则以\textit{Template:Infobox}开头,但也有例外,在中文维基中,类似\textit{Template:电影信息框}等具有语言本地化特色的标签有很多。由此可见,单纯通过词条标签来筛选出信息框模板并不是一个良策。

通过对模板内容进行总结,我们发现信息框模板的源代码中都含有\textit{infobox}字样,该infobox 所涵盖的区域中,包含模板属性定义,可以认为带有infobox信息的Template 为信息框模板。

尽管区分出信息框模板,模板定义的内容格式却不尽相同。根据版式特点,我们将信息框模板归为三类,分别为键值对模板、表格模板与继承模板,图\ref{fig:template-examples}给出三种模板示例,我们将在\ref{sec:property-extraction}节中给出详细介绍。

\begin{figure}[ht]
\centering
    \begin{subfigure}{7.2cm}
        \centering
        \fbox{\includegraphics[width=0.8\textwidth,height=5.6cm]{template-keyvalue}}
        \caption{键值对模板示例}
        \label{fig:template-keyvalue}
    \end{subfigure}
    \hspace{0.01cm}
    \begin{subfigure}{7.2cm}
        \centering
        \fbox{\includegraphics[width=0.8\textwidth,height=5.6cm]{template-inherit}}
        \caption{继承模板示例}
        \label{fig:template-inherit}
    \end{subfigure}
    \vspace{0.01cm}
    \begin{subfigure}{9.6cm}
        \centering
        \fbox{\includegraphics[width=0.8\textwidth]{template-table}}
        \caption{表格模板示例}
        \label{fig:template-table}
    \end{subfigure}
\caption{模板类型举例}
\label{fig:template-examples}
\end{figure}

除了以上三类,数据文件中还出现了{\heiti 重定向模板},内容如图\ref{fig:template-redirect}。该类模板的出现,是因为维基百科经过多次整理与更新,会将相似模板合并,或重新定义模板名称,被删除的模板会重定向到新模板上。2016年2月的中文维基上,\textit{Template:Infobox Film}与\textit{Template:Infobox film} 都重定向到\textit{Template: 电影信息框}。

\begin{figure}[ht]
  \centering
  \fbox{\includegraphics[width=0.75\columnwidth]{template-redirect}}
  \caption{重定向模板内容展示}
  \label{fig:template-redirect}
\end{figure}

{\heiti 属性获取问题:}
维基所提供的数据文件\footnote{https://dumps.wikimedia.org} 中,词条信息框的内容是以{\heiti 模板标签}来组织的,而模板标签与真正展示在网页上的{\heiti 显示标签}不同。模板标签与显示标签的关系在对应语言的信息框模板词条下有定义。如图\ref{fig:film-tl-rl},电影\textit{疯狂动物城}的配乐为\textit{迈克尔·吉亚奇诺},这条信息在词条中的数据文件中是\textit{music=[[麥可·吉亞奇諾]]},而在网页中的显示是\textit{配乐作曲 \ 迈克尔·吉亚奇诺},其中,\textit{music}为模板标签,\textit{配乐作曲}为显示标签,而两种标签的对应关系,在词条\textit{Template:电影信息框}中有所说明(见图\ref{fig:template-infobox-film})。维基百科的这种设计,使模板属性在不同语言上有了统一标签。对于任意一个语言,在设计自己的电影信息框模板时,只需根据模板标签,给出对应的显示标签即可,对多语言百科来说,不失为一种便利的方法。但是间接获得显示标签,以及人为设计的不一致性,都给属性的获取增加了难度。

\begin{figure}[ht]
  \centering
  \includegraphics[width=0.84\columnwidth]{film-tl-rl}
  \caption{电影《疯狂动物城》模板标签与显示标签对比}
  \label{fig:film-tl-rl}
\end{figure}

\section{问题描述}

一个指定的概念$c \in C$所代表的领域下,会有一个属性集合$P(c)=\{p_{i}|i=1,2,...,n\}$ 表征$c$ 中的所有实例$I(c)$的特性。我们定义概念属性$p_{c}= <label>, p(c) \in P(c)$,其中$c$表示该属性的领域信息,只能用来描述领域内的实例$i_{c} \in I(c)$。$label$表示属性标签,属性标签相同与属性相同不等价,即给定$c_i,c_j \in C$,$p_{c_i}.label == p_{c_j}.label\nRightarrow p_{c_i} == p_{c_j}$。

本章的工作是从维基百科中生成概念属性。给定一个维基百科$W$,可以得到一系列模板$T$,模板中定义了描述一类实例的属性集合,对于模板$t \in T$,我们有$P(t)=\{p_{i}|i=1,2,...,n\}$。可以认为,模板代表着一类概念,则有$T \approx C$。基于\ref{sec:template-analysis}中对模板与属性的分析,我们对概念属性的定义需要进行如下补充说明:
\begin{itemize}
\item 维基百科在数据文件中使用的属性标签,与现实在页面的不一致(图\ref{fig:film-tl-rl})。我们将数据文件中使用的标签定义为模板标签(template label),用$tl$表示,展示在页面上的称为显示标签(render label),用$rl$表示。则有$p_{c}.label = <tl, rl>$。
\item 维基百科的模板可能有相近或重复,此时所代表的概念是存在包含关系的。因此严格来说,有$c = \{t_i|i=1,2...n\}$。
\end{itemize}

具体到维基百科,本章工作抽象为:给定一个维基百科$W = <T>$,生成概念集合$C$,其中$c = \{t_i|i=1,2...n\}, c \in C, t_i \in T$,以及概念属性集合$\mathbb{P} = \{P(c_i)| i = 1,2,...,n\}$,其中对于$p_{c_i} \in P_{c_i}$,有$p_{c_i} = <c_i, tl, rl>$。

\section{概念属性生成}
\label{sec:property-extraction}

我们对三类信息框模板分别进行抽取,收集模板标签与显示标签的关系。我们发现,一个属性的显示标签,可能根据模板标签有所变化。比如“公司”概念下的所有者,在模板标签为owner时显示\textit{Owner},在显示标签为owners时,显示\textit{Owner(s)}。抽取过程中,所有模板标签与显示标签的关系都会保留下来,我们以$<T, tl, rl>$ 的格式保留一组属性记录。

{\heiti 键值对模板($T_{kv}$)(图\ref{fig:template-examples}a):} 这种类型的模板,其显示标签与模板标签在infobox中以

\begin{center}
\verb"label=显示标签"

\verb"data=模板标签"
\end{center}
的格式存在。一个模板中,显示标签与模板标签的关系可能是一对一、一对多、多对多的,即一个属性根据不同的模板标签,可能会显示不同的标签在页面上。

{\heiti 表格模板($T_{table}$)(图\ref{fig:template-examples}b):}这种类型的模板以类似表格的形式对属性进行编排,抽取方法与键值对模板不同,根据表格行列思想,通过分析$\langle td\rangle$等标签抽取。

{\heiti 继承模板($T_{inherit}$)(图\ref{fig:template-examples}c):}有半数左右的模板存在继承关系,即其模板标签的表示含义与父类模板相同,标签对应关系要结合父类模板来挖掘。继承模板可能继承自键值对模板、表格模板,甚至重定向模板。

{\heiti 重定向模板($T_{redirect}$)(图\ref{fig:template-redirect}):} 这类模板虽然没有实质内容,也不建议继续使用,但其重定向关系对于分析实例对模板的应用情况至关重要,编辑者可能在编写信息框时用的旧版模板标签。图\ref{fig:template-redirect-examples}给出不同词条对同一模板的不同编辑方式。三个不同的电影词条分别可能用到\textit{Template:Infobox Film},\textit{Template:Infobox film},\textit{Template:电影信息框}。如果不考虑重定向模板标签,用前三类的抽取结果直接匹配,会大大影响已抽取模板对词条的覆盖率。因此重定向模板不能弃之不理。

\begin{figure}[ht]
  \centering
  \includegraphics[width=0.8\columnwidth]{template-redirect-examples}
  \caption{重定向模板使用情况举例}
  \label{fig:template-redirect-examples}
\end{figure}

\subsection{结果统计}

基于以上四类信息框模板,我们对2016.03版本的英文维基与2016.02 版本的中文维基分别抽取,表\ref{tab:wiki-infobox-statistic}是给出了两中语言维基的统计信息。

\begin{table}[htb]
    \centering
    \caption{中英文维基百科词条信息统计}
    \label{tab:wiki-infobox-statistic}
    \begin{tabular}{cccc}
    \toprule[1.5pt]
    {\heiti 维基百科} & {\heiti 词条数} &  {\heiti 含有信息框的词条数及比例} & {\heiti 使用模板数} \\\midrule[1pt]
    英文维基 & 5,138,426 & 2,344,305(45.6\%) & 4491 \\
    中文维基 & 863,918   & 261,045(30.2\%)   & 2317  \\
    \bottomrule[1.5pt]
    \end{tabular}
\end{table}

得到各类型信息框模板数量如表\ref{tab:infobox-template}所示。

\begin{table}[htb]
  \centering
  \caption{各类型信息框模板数量}
  \label{tab:infobox-template}
  \begin{tabular}{cccccc}
  \toprule[1.5pt]
      {\heiti 维基百科} & {\heiti $T_{kv}$} &  {\heiti $T_{table}$} & {\heiti $T_{inherit}$} & {\heiti $T_{redirect}$} & {\heiti 总数}\\\midrule[1pt]
      英文维基 & 1910 & 1901 & 1897 & 4076 & 9784\\
      中文维基 & 840  & 878  & 968  & 956  & 3642\\
  \bottomrule[1.5pt]
  \end{tabular}
\end{table}

在抽取过程中,我们统计出以\textit{Template:Infobox} 标识的模板占所有英文信息框模板的91\%,中文的63.5\%。可见仅仅通过模板词条来确定信息框模板是不可行的,我们对模板更为精细的解析方法更有助于全面获取概念属性信息。

因为数据杂乱等原因,解析过程中无法成功获取所有模板中的信息,我们以常用模板为优先,并以词条覆盖率90\%以上为目标结束概念属性生成任务。成功解析的属性信息如表\ref{tab:render-label}所示,其中不包括对重定向模板。

\begin{table}[htb]
  \centering
  \caption{概念属性信息生成结果}
  \label{tab:render-label}
    \begin{tabular}{ccccc}
    \toprule[1.5pt]
      {\heiti 维基百科} & {\heiti 模版数} & {\heiti 模板属性数} & {\heiti 属性/模板}  &{\heiti 模板标签/属性} \\\midrule[1pt]
      英文维基 & 3819 & 104741 & 27.45 & 1.27  \\
      中文维基 & 1895 & 44797  & 23.64 & 1.34  \\
    \bottomrule[1.5pt]
    \end{tabular}
\end{table}

我们在\ref{tab:coverage}中给出当前生成的概念属性对维基百科的覆盖程度。实际使用的模板名称与模板正式名称的匹配率较低,需要经过一系列转化,包括:大小写转换、去除多余空格、重定向模板转换等。
模板覆盖率为抽取出的模板数量与所有词条使用的全部模板数量的占比,模板属性覆盖率为所有抽取出的模板属性对所有词条的所有属性数量的占比,词条覆盖率为能用抽取模板分析的词条的百分比,我们获取的概念属性可以处理90\%以上的维基词条的信息框。

\begin{table}[htb]
  \centering
  \caption{结果覆盖率统计}
  \label{tab:coverage}
    \begin{tabular}{cccc}
      \toprule[1.5pt]
      {\heiti 维基百科} & {\heiti 模板覆盖率} & {\heiti 模板属性覆盖率}  & {\heiti 词条覆盖率} \\\midrule[1pt]
      英文维基 & 87.6\% & 27\% & 96.7\%  \\
      中文维基 & 77.8\% & 33\% & 90.3\%  \\
      \bottomrule[1.5pt]
    \end{tabular}
\end{table}

表\ref{tab:template-examples}与表\ref{tab:template-property-examples}给出了概念生成结果举例,其中
表\ref{tab:template-examples}给出了常用模板举例,表\ref{tab:template-property-examples}给出了三个常用中文模板的部分属性举例。

\begin{table}[htb]%建议把模版名中的infobox删掉
  \centering
  \caption{常用模板统计}
  \label{tab:template-examples}
    \begin{tabular}{llll}
    \toprule[1.5pt]
    \multicolumn{2}{c}{\heiti 英文维基}   & \multicolumn{2}{c}{\heiti 中文维基} \\
       模版名称&使用词条&模版名称&使用词条 \\\midrule[1pt]
       Infobox settlement   & 253276 & Infobox Settlement & 34717 \\
       Infobox person       & 158481 & 艺人 & 14047 \\
       Infobox football biography & 127402 & Infobox Planet& 12163 \\
       Infobox film         & 102863 & 文物保护单位& 10362 \\
       Infobox officeholder & 86698 & Infobox officeholder& 7501\\
       Infobox album        & 77436 & Infobox French commune& 7297\\
       Infobox musical artist & 74469 & 电影信息框& 6787 \\
       Infobox company      & 46646 & Infobox person& 6414 \\
%       Infobox NRHP         & 42385 & 电视节目信息框& 5958\\
%       Infobox single       & 40578 & Infobox Company& 5138 \\
    \bottomrule[1.5pt]
    \end{tabular}
\end{table}

\begin{table}[htb]
  \centering
  \caption{常用模板属性举例}
  \label{tab:template-property-examples}
  \begin{tabular}{llllll}
      \toprule[1.5pt]
        \multicolumn{2}{c}{\heiti 艺人}  & \multicolumn{2}{c}{\heiti Planet} & \multicolumn{2}{c}{\heiti 电影}\\
        模板标签& 显示标签 & 模板标签& 显示标签& 模板标签& 显示标签\\ \midrule[1pt]
        Alias & 昵称      & discovered & 发现日期      & budget & 预算 \\
        hometown& 籍贯    & discoverer & 发现者        & director & 导演\\
        URL& 网站         & density & 平均密度         & producer& 监制 \\
        child& 儿女       & mass & 质量                & writer & 编剧\\
        children& 儿女    & inclination & 轨道倾角     & starring & 主演\\
        country & 国籍    & named\_after& 命名依据     & music &  配乐作曲\\
      \bottomrule[1.5pt]
  \end{tabular}
\end{table}

%\section{概念合并}
%
%最后我们有xx个概念,共xx个属性
%
\section{本章小结}

本章主要从维基百科中生成概念属性。我们将维基模板相关的领域视为概念,其中的属性视为该概念下的属性,这种方法同时可以解决属性多义问题,即认为同一领域下的属性没有歧义性。

为了获得尽可能多的概念属性,我们对维基百科模板和属性进行了细致的分析,将模板分为键值对模板、表格模板、继承模板与重定向模板,分别进行处理,将属性赋予模板标签与显示标签两种信息,这些工作,有助于他人对于属性和模板有更深入的理解。

