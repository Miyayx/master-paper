\thusetup{
  %******************************
  % 注意:
  %   1. 配置里面不要出现空行
  %   2. 不需要的配置信息可以删除
  %******************************
  %
  %=====
  % 秘级
  %=====
  secretlevel={秘密},
  secretyear={10},
  %
  %=========
  % 中文信息
  %=========
  ctitle={跨语言知识图谱中的概念属性生成与对齐},
  cdegree={工程硕士},
  cdepartment={计算机科学与技术系},
  cmajor={计算机技术},
  cauthor={李明洋},
  csupervisor={李涓子教授},
  %cassosupervisor={陈文光教授}, % 副指导老师
  %ccosupervisor={某某某教授}, % 联合指导老师
  % 日期自动使用当前时间,若需指定按如下方式修改:
   cdate={二〇一六年四月},
  %
  % 博士后专有部分
  cfirstdiscipline={计算机科学与技术},
  cseconddiscipline={系统结构},
  postdoctordate={2009年7月——2011年7月},
  id={编号}, % 可以留空: id={},
  udc={UDC}, % 可以留空
  catalognumber={分类号}, % 可以留空
  %
  %=========
  % 英文信息
  %=========
  etitle={Concept-Property Generation and Alignment in Cross-lingual Knowledge Base},
  % 这块比较复杂,需要分情况讨论:
  % 1. 学术型硕士
  %    edegree:必须为Master of Arts或Master of Science(注意大小写)
  %             “哲学、文学、历史学、法学、教育学、艺术学门类,公共管理学科
  %              填写Master of Arts,其它填写Master of Science”
  %    emajor:“获得一级学科授权的学科填写一级学科名称,其它填写二级学科名称”
  % 2. 专业型硕士
  %    edegree:“填写专业学位英文名称全称”
  %    emajor:“工程硕士填写工程领域,其它专业学位不填写此项”
  % 3. 学术型博士
  %    edegree:Doctor of Philosophy(注意大小写)
  %    emajor:“获得一级学科授权的学科填写一级学科名称,其它填写二级学科名称”
  % 4. 专业型博士
  %    edegree:“填写专业学位英文名称全称”
  %    emajor:不填写此项
  edegree={Master of Engineering},
  emajor={Computer Technology},
  eauthor={Li Mingyang},
  esupervisor={Professor Li Juanzi},
  %eassosupervisor={Chen Wenguang},
  % 日期自动生成,若需指定按如下方式修改:
  edate={April, 2016},
  %
  % 关键词用“英文逗号”分割
  ckeywords={属性模板, 属性对齐, 知识图谱, 实体链接},
  ekeywords={Template, Property Alignment, Property Matching, Knowledge Base, Entity Linking}
}

% 定义中英文摘要和关键字
\begin{cabstract}
随着语义网的快速发展,单语言知识库已经不能满足现今全球知识融合的需求,跨语言知识图谱逐渐受到人们的重视。虽然DBpedia,YAGO等大型语义数据集提供了丰富的多语言信息,中文知识依然稀少
%,主要因为基于维基百科构建的知识库受维基中各语言知识量不平衡的局限
。为了实现中文知识与全球知识库的融合,跨越中文资源匮乏的障碍,构建一个大规模的中英文跨语言知识图谱势在必行。

在构建跨语言知识图谱的过程中,本体构建与对齐至关重要。概念与属性作为本体的主要元素,起着描述实例、关联实例的作用。一个概念领域内的属性集合构成了描述实例的框架模板。如何定义并获取概念属性关系,另其对实例的描述更准确?如何在跨语言环境下将概念属性关联?这些都是跨语言知识库需要解决的问题。

本文针对概念领域内的属性,展开研究,首先通过分析维基百科的信息框,生成一系列概念属性集合,之后在异构百科下进行概念属性跨语言对齐,最后将结果应用于跨语言知识图谱XLore上。

维基百科在信息框的组织中使用模板进行规范,通过对维基信息框模板进行详尽的分析,对模板归类并制定抽取方法,最终生成大量概念属性信息,可用来描述90\%以上的维基词条,为属性的进一步研究提供了基础数据。

为解决中英文知识数量的差异问题,引入大型中文百科数据,提出异构百科下的跨语言属性对齐框架。该框架首先提出概念属性生成方法,从概念上对齐;再次通过同语言对齐与同义词合并,解决百科差异性;然后通过各百科互对齐关系,首先获得一批高质量的跨语言对齐种子;最后提取文本、语义、分布上的特征,训练二分类器,在概念下进行跨语言链接的判断,发现更多新链接。

跨语言属性对齐结果应用在基于异构百科构建的跨语言知识库XLore中。为了直观地观察语义知识,基于XLore搭建展示系统,将知识进行多角度多语言的展示与查询。 
在知识库的应用方面,提供了基于XLore的实体链接接口。该接口支持中英文学术词汇、通用实体链接,并提供对应需求的实体消歧算法,最终应用在学术网站与新闻研究领域。

\end{cabstract}

% 如果习惯关键字跟在摘要文字后面,可以用直接命令来设置,如下:
% ckeywords={属性模板, 属性对齐, 知识图谱, 实体链接},

\begin{eabstract}
As Semantic Web develops, monolingual knowledge base cannot follow the growing requirement of global knowledge fusion. Cross-Lingual Knowledge Base (CLKB) is paid more attention. Although some famous semantic datasets, such as DBpedia and YAGO, provide abundant language links, Chinese Knowledge is still too less to enrich global knowledge base. It is very necessary to overcome the problem of Chinese resource starvation and construct a large-scale Chinese-English Cross-Lingual Knowledge Base.

Ontology Building and Mapping are crucial to construct a CLKB. As the most important elements in Ontology, \textit{Concept} and \textit{Property} take up the job of describing instances. In a concept domain, property set comprises the description template of instances. How to define the concept-property and take advantage of them? How to align the cross-lingual concept-properties? All are problems remain to be solved when building CLKB.

This paper pursues research for concept-property. Firstly a series of concept-properties are generated by analyzing infobox templates in Wikipedia. Then, the cross-lingual concept-properties from heterogeneous online wikis will be mapped. Finally, the alignment result will be used in a CLKB, named XLore.

Wikipedia utilizes Infobox Template to organize the content in article infobox. According to the analysis of Infobox Template, some particular extraction methods are custom-made to help generate massive concept-properties. The result covers more than 90\% article infoboxes in Wikipedia, and supplies basic data for further research in concept-property.

To solve the imbalance between Chinese and English knowledge quantity, a large-scale Chinese online wiki is imported, and a cross-lingual property matching framework based on heterogeneous wikis is proposed. The framework includes four steps. Firstly, it presents a method to generate concepts and properties. Secondly, it aligns properties in monolingual wikis and merges synonymous properties. Thirdly, cross-lingual seeds of high quality are generated for the later alignment. At Last, a binary classifier based on textual, semantic and popular features is trained for cross-lingual alignment prediction and new cross-links discovery.

The cross-lingual property matching result is utilized on a CLKB named XLore, which is based on heterogeneous wikis. To obtain intuitionistic observation of the XLore, a system for knowledge presentation is set up, which provides varied methods for showing and querying knowledge. In the aspect of application, an entity linking API is created based on XLore. The API supports term and normal entity linking in both Chinese and English, and implements disambiguation algorithm. The API is used on academic website and news domain finally.

\end{eabstract}

% \ekeywords{\TeX, \LaTeX, CJK, template, thesis}
%ekeywords={Template, Property Alignment, Property Matching, Knowledge Base, Entity Linking}
