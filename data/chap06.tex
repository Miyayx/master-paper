\chapter{总结与展望}
\label{cha:conclusion}
本章对本文涉及的工作内容进行了概括与总结,并从属性对齐、知识库构建与应用角度出发,提出待改进之处。

\section{现有工作总结}
本文对维基百科与百度百科的信息框模板与属性进行了深入的研究,涉及到跨语言与跨百科两部分问题。通过对数据细致入微的观察,本文通过模式匹配、相似度计算等方法,生成维基百科与百度百科中的概念与对应属性集合,并计算出异构百科中的跨语言信息框属性对齐关系。同时,将属性对齐结果应用于跨语言知识库XLore的构建中。

首先,在同百科跨语言研究上,我们对维基百科的信息框编辑机制进行了详细的分析,发掘出信息框模板组织方式与属性的定义模式,并基于此做了两项工作:1. 维基百科显示标签的抽取;2. 根据信息框模板提供概念,并生成概念属性。最终获得的结果覆盖90\%以上词条,为后续异构百科的属性分析提供了重要信息。

其次,在异构百科跨语言研究过程中,我们面临百度百科中属性的歧义与多表示问题。从模板得到启发,我们为以维基模板概念为指导,为百度属性划分领域,在领域范围内对齐属性,避免属性多义问题。另在同语言百科对齐同时查找属性同义表达。以中文维基为桥梁,首先获得一批准确的英文维基与百度百科的对齐关系,以此为种子集合,提出基于文本、语义、分布三类特征,训练二分类模型,实现异构百科下的跨语言链接。

属性的分析结果主要应用于跨语言知识库XLore。该知识库基于中英文维基百科、百度百科、互动百科四个数据源构建,抽取概念、属性、实例,使用跨语言链接融合中英文知识。最终结构化数据转换为标准的RDF三元组。基于该知识库数据,我们建立了XLore.org展示系统,提供知识的可视化与查询。同时,从知识库应用角度,开发了实体链接接口,支持通用领域与学术领域的实体识别与实体链接,返回相关语义信息。该接口应用到学术术语词汇查询、新闻文本分析等领域上。

总体来说,本文的重点在概念属性分析上,并辅助跨语言知识库的构建,该工作对于知识库Schema定义、中英文知识平衡等有着十分积极的意义。此外,本文所提供的维基数据的分析、跨语言知识库建立的流程,对于从结构化数据中构建知识库有很大借鉴意义。

\section{未来工作展望}
在信息框属性的分析因属性粒度小特征少,数据不归整等原因,是一项繁杂的工作。通常来说,分辨属性的最佳特征为属性值,但值文本的复杂性增加了特征提取难度与特征准确性,因此还需要其他特征的辅助,文章、模板、领域等等元素可以作为提取特征的候选,不过在本文工作中并未被充分利用。此外,跨语言对齐受翻译正确率、有限的已知跨语言信息等因素的限制,也需要考虑更独立的方法。

归结起来,本文在信息框属性方面,提出以下几个待提高点:
\begin{enumerate}[1)]
\item {\heiti 相似属性合并:} 属性存在多表示问题,我们提出寻找相似属性的任务来提高被处理属性的召回率。为了减小对后期跨语言对齐的影响,我们舍弃了召回率,保留了基于同语言对齐而提取出的更为精准的属性多表现形式。然而,同语言相似属性的查找依然是属性对齐过程中一个亟待解决的问题。如何通过同语言间的关系,找出更多、更准确的属性表现形式,是我们需要继续研究的工作。
\item {\heiti 抽取更多跨语言属性关系:} 基于维基百科开展的属性对齐研究有很多,大部分是对属性值进行细致地处理后,通过值对齐进而得到属性对齐关系。本文对属性值的分析还不到位,比如属性类别还可以进一步划分出日期、列表等类型,文本属性值还需要进行清理。
\item {\heiti 获取更多跨语言信息:} 跨语言属性关系与特征的抽取都基于跨语言词条,维基百科的语言链接数量有限,会成为属性对齐分析的瓶颈。此外,跨语言信息对构建跨语言知识库也很重要。如何扩展跨语言链接集合,是很有意义的一项研究任务。使用词条与属性链接相互增加迭代的方式,同步扩展两者的跨语言对齐关系,也有必要一试。
\item {\heiti 属性特征提取:} 本文主要从文本、语义等方面提取特征,事实上,结构信息、属性关联度等都可能有一定帮助,充分利用百科结构化信息,有利于我们标识属性,生成更多跨语言链接。
\end{enumerate}

在知识库构建方面,本文所涉及的跨语言知识库XLore融合了四个百科的数据,在对齐数量、数据正确性方面都还有待提高,许多研究点如分类体系的形成、信息框填充、百科缺失链接处理等都可进行深入思考。

知识库的质量与其可用性密切相关,在应用方面,如何广泛、有效利用知识库也值得深究。辅助优化搜索、提供关联信息等都是可以尝试的应用。此外我们希望,涵盖了通用领域知识的XLore,也可以在特定领域中大显身手,面向学术领域的实体链接,即为一个尝试。具体来说,给定一个概念,能在知识库中直接划分出该领域的相关概念、实例、属性,形成领域知识,这就能很好地服务于限定领域的应用。

对于目前已存在的实体链接应用,本文提供了基础解决方案与框架,然而这条路还很长,我们还需要从识别实体文本、精确消歧、解决无信息实体,甚至提高查询响应度等方面,提高接口的可用性。

本章提出了工作需要进行的改进以及可能进行的突破,这些是我们未来工作的方向。为打造一个高质量、实用的跨语言知识库,我们依然要不懈努力。

