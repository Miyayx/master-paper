\chapter{大规模中英文跨语言知识库XLore的构建}
\label{cha:cross-lingual-knowlege-base}

\section{本章引论}

跨语言知识库对全球知识的共享有着很大贡献。然而,中英文的跨语言知识库却很少,主要有以下几点原因:1)相对于丰富的英文知识,中文知识极度匮乏2)现有中英文跨语言对应关系的缺乏 3)本体语义分类树中,上下位关系存在错误。 为解决以上问题,我们在本章提出了一个构建中英文跨语言知识库(Cross-lingual knowledge base)的方法与流程,并将该知识库命名为XLore。具体来说,XLore囊括了英文维基,中文维基,百度百科与互动百科,四个在线百科的知识,以此来均衡中英文知识数量,同时,利用一个跨语言链接发现方法扩充跨语言链接集合,并引入了一个上下位关系判断,对分类树进行剪枝。

本章将详细介绍构建中英文知识库XLore的过程与相关技术, 内容按照如下方式组织:\ref{sec3:datasource}介绍数据源、3.2节详细介绍语义数据抽取过程、3.3节叙述跨语言知识融合过程、3.4节展示构建结果数据、3.5节对本章进行总结。


\section{数据源}
\label{sec3:datasource}

以构建大规模中英文知识图谱为目标,对数据来源有以下几项要求:

\begin{itemize}
\item  {\heiti 知识覆盖度广:} 丰富的。。。。
\item  {\heiti 数据质量高:} 数据应该经常有专业团队维护与实时更新,保证知识可信度高,有助于维护知识图谱的权威性;
\item  {\heiti 数据结构化好:} 结构化的数据大大减少了构建的工作量,方便进行各数据源的整合。当前大部分开源知识图谱,如DBpedia,都基于维基百科构建,高效地利用其结构化信息,形成一个更为精细的知识源。当然,也有一些研究致力于从纯文本中构建知识图谱,比如ProBase从大量网页中抽取概念、实例、属性、关系,构建成一个大型本体。但文本抽取方法受过多杂质干扰,知识准确性验证也需要耗费更多精力,有很大难度。
\item  {\heiti 拥有跨语言特性:} 准确的跨语言链接是构建跨语言知识图谱来的核心,然而,当前跨语言知识库,尤其是中英文跨语言知识库极为匮乏,知识分布也不均匀,因此人工确定的、精准的跨语言链接极为重要,不仅可以在构建过程中起到链接作用,还可以起到种子的作用,对进一步抽取、识别更多跨语言对至关重要。
\end{itemize}
综合以上需求,选择四个百科数据作为构建数据源,在本小节中分别介绍。

\subsection{百科介绍}

目前为止,维基百科是最大的一个知识存储库。它建立于2001年,截止到2016年,以及涵盖了288种语言下的xxxx万条词条。其中,英文词条占xx%。当前,有诸多单语言和多语言的知识库基于维基百科构建,包括xxx单语言知识库,xxx跨语言知识库。

图XX为截止到2016年3月,维基百科10个主要语言的词条数量。可以看到,不同语言的词条数量分布极不平均,而中文知识也仅占英文知识量的xx%。因此,为构成中英文跨语言知识库,还需要利用其他中文知识库来扩充中文知识。

在当前大规模的中文单语言百科中,百度百科与互动百科是目前国内知识最为丰富、认可度最高的两个网络百科全书平台。与维基百科相同,他们都由人工编辑,人工校验,并实时更新,规模一直在快速增长。然而,不同于维基百科的数据开放性,百度百科与互动百科的数据是封闭的,不提供获取最新数据资源的途径。

互动百科于2005年创立至今,已涵盖了由1047万个专业认证志愿者编辑的1430万多个词条。

百度百科于2006年创立,截止2016年1月,词条数目已超过1313万。

\subsection{页面元素介绍}

四个在线百科的词条页面在布局上很相近。通常,一个词条页面可以贡献两种带有潜在语义信息的元素,分别为所属分类信息与词条内容信息。分类信息包含负分类、自分类及其上下位关系信息。 图xxx是互动百科分类页面的截图。

一个词条页面描述了一个显示中的实例。这些实例的丰富的信息由一个或多个经过专业认证的编写者们创建或修改,一般其信息内容都一定的可考证依据。另外,一篇词条可能归属于一个或多个分类。词条自带的实例内容、其与其他词条的关系,以及它所属的类别,都在构建知识库时起着重要作用。一般来说,一篇词条页面中,有五种元素可以被利用:

{\heiti 标题:} 词条标题,即其所表征实体的名称,可以作为区分不同实体的表识,其内容在同一百科中是唯一、不可重复的。维基百科、互动百科、百度百科都对重名的不同实体做消歧处理,并在标题上表现出来,比如。。。

{\heiti 摘要:}摘要是实体的一端剪短的简介,通常包含该实体的重要信息。一般是页面正文的第一段话。

{\heiti 信息框:}一部分词条中带有信息框。信息框是一个表格形式,内含属性-属性值对,带有实体重要的属性信息。从信息框中可抽取主语-属性-属性值的三元组格式的结构化信息,其中主语即为该词条所表征的实体。

{\heiti 链接:}词条文本中的超链接部分,是另一个词条的入口,点击即可进入对应词条的页面。它表征该词条与百科中其他词条的关系,是构成知识库中实体与实体关系的重要信息。
分类:一个词条所属分类一般在该词条页面的底部一行显示,已标签或超链接的形式渲染。一个词条可能归属于一个或多个分类,表征知识库中的is-a关系。

图XX是一篇中文维基的词条页面

需要提到的一点:维基百科中页面内容的组织是遵从于维基模版制成的。此模版对词条编辑时,所要遵循的内容、结构等信息进行了约束。比如xxx,xxx。信息框的填充也从模板中产生。比如:词条 星际穿越根据模板Infobox film来编写,按规定来说,电影词条的信息框信息都应根据此模版来编辑。信息框模板带有电影领域公认的、常用的属性集合。

除此之外,维基百科中,大部分词条的页面左边栏中,还提供了跨语言链接,该链接指向其他语言下该实体的词条页面。利用这些人工构建的跨语言信息,我们可以获得初步的跨语言链接集合。


\section{语义信息抽取}
\label{sec3:extract}

语义信息的抽取,旨在以在线百科页面信息为输入,获得一个结构化的数据集,为知识库的构建做准备。具体来说,我们从分类树中抽取概念信息,从词条中获取实例信息,从词条的信息框以及其遵循的信息框模版中,得到属性信息。

\subsection{概念抽取}

概念是指一些相似实例的所属类型。举例来说:电影实例《星际穿越》(Interstellar)的概念即为电影(Movie)。通常来说,一个概念应该有多个父分类与多个子分类,该概念与父分类的关系,以及子分类与该概念的分类为SubClassOf关系(出处?). 所有概念一起构成一个分类树,作为本体的骨架。

在百科中,一个分类集合了几个词条,且分类既有父分类,也有子分类,与本体中的概念相似,因此我们可以从百科的分类系统中抽取出概念来。

但是,整个分类系统不能直接被构建成本体中的taxonomy,原因如下:

维基百科中有一些辅助性的分类,用来组织特定领域的词条或分类页面,比如: List of artists 或者 food template

因为人为编辑的随意性,会有一些小分类也被编入其中,这些分类可能只包含一个或两个词条。根据概念的定义,这些分类对词条类型的代表性很小,不适于被看作概念。

为了得到一个更精准的概念体系,我们对已有的维基分类系统进行了如下的剪枝操作:

删除维基百科中的列表分类(list of 。。。)以及模版分类(template。。。)

删除只包含一到二个词条的分类小分类

这些清洗工作在概念抽取的工程中就一并进行了。留下的分类形成一个原始的概念层次体系。但是这个体系中仍然存在不正确的关系对。举例来说,Tsinghua University 并不是 Haidian District的一个实例,只是相关而已,这样模棱两可的关系在维基百科的分类体系与分类和实例的上下位关系中举不胜举,我们在第xxx章会对这个概念体系进行剪枝操作,力求得到一个更客观、准确的上下位关系体系。

\subsection{属性抽取}

实体的属性是指实体的某种特性。表征该实体与其他实体,或属性值的关系。

从属性值来区分,属性分成了两种类型:对象属性与数值属性。其中对象属性的值是一个实体,比如“导演”;数值属性的值是一个字符串,比如“出生日期”。

从属性性质区分,属性分为了通用属性与信息框属性。实体在百科中的公有特性称为通用属性,包括名称、摘要和url地址。通用属性都是数值属性。信息框属性来自词条的信息框,比如电影实例的信息框中的上映时间(release date),导演(directed by)。信息框属性是对象属性还是数值属性,取决于属性值。通常,属性值是普通文本的是数值属性,而属性值是对另一个实体的引用的,标志该属性是对象属性。举例来说,属性“上映时间”被定义为数值属性,因为它的属性值是一个描述日期的字符串。与此同时,属性“导演”被定义为对象属性,因为它的属性值指向指导这个电影的导演对应的词条。

在从词条信息框中抽取信息框属性时,存在一些挑战:

\begin{itemize}
\item 在维基百科里,在网页上显示的属性标签(本文称之为显示标签),与维基公开发布的数据文件中的属性标签(本文称之为存储标签),是不一致的。具体来说,维基百科使用信息框模版来规范编辑者的信息框填充行为,该模版定义了某领域的词条可能具有的性质,并自定义一套存储标签与显示标签的对应关系。以“星际穿越”实体为例,图xx展示了显示标签与存储标签相互对应的例子,其中第一个子图是显示在网页上的信息框,第二个子图是数据文件中信息框部分的数据展示。可以看到,显示标签与数据集中的存储标签是不一致的,这为我们抽取正确的信息属性标签,并形成跨语言属性链接提高了难度。
\item 一些属性标签中存在特殊字符。比如维基百科用短线- 或 点”” 表示子标签。举例来说,“population”属性含有子属性“-Density”与“-Urban”。除此之外,冒号,星号,多余空格等,都会由于编辑失误、个人习惯等原因,出现在百度百科或互动百科的属性标签中。
\end{itemize}

为解决以上问题,我们添加了显示标签抽取模块与标签清理(过滤?)模块。

\begin{itemize}
\item 显示标签抽取:利用维基百科的模版信息,找出存储标签与显示标签的对应关系,用显示标签替换存储标签。举例来说,电影词条的信息框的编写一般遵循电影信息框模版(Template:infobox Film),如图XX的最右子图所示,存储标签与显示标签都来源于此模版。对显示标签“导演”来说,它的存储标签为“director”。找到这个对应关系,将其带入维基数据文件中,即可得到与网页信息框显示一致的键值对。
\item 标签清理: 对各百科的显示标签进行清洗,删除如星号、冒号、空格等多余字符,与标准的标签合并为同一个属性。
\end{itemize}

\subsection{实例抽取}

在百科中,一个词条文档是对世界上唯一实体的描述。因此我们将词条对应的实体看作知识库中的实例(instance)。

维基百科中包含大量自己定义的辅助性页面,会混入抽取的词条结果中。在抽取过程中,我们通过定义词条标题模式,将说明型(比如Media:,Help:)、结构型(比如Template:)以及列表型(List of。。)等带有维基百科特色的词条删除,只保留实体描述型词条。

每个词条页面都包含分类关系、属性信息。以图xxx中,电影星际穿越(Interstellar)的词条页面为例,该词条所属的类别,以超链接标签的形式罗列在页面底部。我们认为这是该实例所具有的概念,即美国科幻片(American science fiction films)是星际穿越(Interstellar)的一个概念(Concept),存在 Interstellar <InstanceOf> American science fiction films的关系。另一方面,我们从词条标题抽取出标题属性的值,从文章第一段抽取出摘要属性的值。 所有信息框属性与属性值从信息框中提取出来。另外,通过抽取文本中的超链接,我们还获得了实例间的引用信息Li(a), 比如华纳兄弟(Warner Bros.) 

实例抽取过程中,我们收获了两种数据信息,分别是实例及其特征信息,包括Ti(a),ab(a),INfobox。。。。 以及上下位关系信息。

\subsection{跨语言知识库构建}

利用现有的结构化数据构建跨语言知识库,我们做了如下4步

1.  收集已知的、可以映射两种语言下的同一实体的跨语言链接,以此为种子,寻找更多跨语言链接,增加链接数量;
2.  整合四个百科中的知识,即利用跨语言链接,将四个百科中代表同一实体的概念、属性、实例找出,合并成一个;
3.  通过概念、实例以及上下位关系,获得分类体系,对其进行关系正确性的检测,从而获得更精准的分类体系;
4.  将实例、属性挂载到分类体系下,形成完整的知识库。

\section{跨语言链接}
\label{sec3:cross-lingual}

维基百科已经人工编辑了跨语言链接,截止到2016年3月版本,维基百科的中英文跨语言链接有xxx个,占英文词条总数的xx%。 通过抽取这些链接,我们可以在概念与实例下,构成一个初始的跨语言链接集合。

之后,我们运用[xx]中提到的语言非依赖方法,扩展链接集合。该方法提出xxx,xxx,xxx的假设,提取出对应特征集合,构建连锁因子图,最终获得基于英文维基百科与百度百科的215thousand个跨语言链接。准确率达到85.5%,召回率为88.1%。

利用该方法,我们以xxxx个维基百科中英文跨语言链接为种子集合,抽取出xxx个维基-百度链接。

然而,属性没有可直接获取的跨语言链接。我们利用维基模版,获取属性的跨语言链接。具体来说,分为以下三步抽取:
\begin{itemize}
\item 1.  基于跨语言模版对齐:对于已对齐的跨语言信息框模版,找出其中存储标签一致的中英文显示标签,构成跨语言属性对,即Te与Tz为跨语言对齐对,给出Te中的pe,与Tz中的pz,如果tle==tlz,则<dle, dlz>是跨语言链接对。
\item 2.  基于跨语言实例对齐:对于已对齐的词条,抽取其信息框模版,找出其中存储标签一致的中英文显示标签,构成跨语言属性对;
\item 3.  基于属性值对齐:对于已对齐的词条,分析其信息框中的属性-属性值,如果两个中英文属性类型都为对象属性,且属性值指向同一个实体;这两个属性构成跨语言属性对。
\end{itemize}

分别获得跨语言属性链接 \ref{tab:property-matching}

\begin{table}[htb]
  \centering
  \caption 跨语言属性链接
  \label{tab:property-matching}
  \begin{minipage}[t]{0.8\textwidth} % 如果想在表格中使用脚注,minipage是个不错的办法
    \begin{tabularx}{\linewidth}{X|X|X|X|}
      {\heiti 基于跨语言模版对齐} & {\heiti 基于跨语言实例对齐} & {\heiti 基于属性值对齐} & {\heiti 总数(删除重复)}  \\\midrule[1pt]
      1 & 1 & 1 & 1 \\
      \bottomrule[1.5pt]
    \end{tabularx}
  \end{minipage}
\end{table}
            
为了提现跨语言特性,我们将来自四个百科中表示同一实体的概念、实例、属性合并成同一个,并用唯一id标识。以实例为例,我们遵循以下步骤合并实例:
\begin{itemize}
\item 1.  对于从中文维基百科抽取的实例,从百度百科与互动百科中,通过词条标题,找到表征同一实体的实例;
\item 2.  对于从互动百科中抽取的实例,从百度百科(不包含中文维基百科)中,通过词条标题,找到表征同一实体的实例;
\item 3.  其他只存在与一个中文百科中的实例,将其看作一个独立实例。截止到该步,我们已经整合了三个中文百科中的实例;
\item 4.  对于跨语言链接Lz(a), 查找在CL中是否存在<Lz(a), Le(a)>. 如果有,将这两个中、英文实例标识成同一个实例;
\item 5.  对于没有跨语言链接的中英文实例,单独编号。
\end{itemize}

经过以上五步,我们获得了中英文融合后的,有唯一标识的实例集合。其中,跨语言的知识信息,如标签、分类等,会通过@en与@zh区分成两类,形成两类知识信息,跨百科知识信息,如摘要、源URL等,会通过@enwiki, @zhwiki, @baidu, @hudong 区分成四类。

合并四个百科的概念与属性的过程,与实例过程一致。

\section{分类系统剪枝}

因为人工编辑的不规范,百科里分类系统中的上下位关系,不可避免地会有很多错误,比如。。。  为解决这个问题,我们引进[xx]中的方法,从抽取出的subCategoryOf和articleOf关系中,判断出正确的subClassOf与instanceOf关系,从而获得更精准的分类体系。具体来说,该方法抽取了基于文本与基于结构的特征,为每个概念与实例构成一个特征向量,通过一个3000数量的种子集合,训练出一个基于逻辑斯蒂回归模型的二分类器,并通过添加准确度高的预测结果,重新训练模型,实现过程迭代。除此之外,预测结果还会经过跨语言信息的检测。

剪枝后的理想状态下的分类体系应该是树状的,其中边、结点与叶子结点分别代表语义关系、概念与实例。但是,因为在删除不正确语义关系的过程中,中没有将完整性考虑在内,实际的结果应该是森林形状的。

为了语义关系的完整性,我们保留了剔除的语义关系,并为其定义了两种新关系,来表征数据的相关性:为实例-概念关系relatedClassOf关系,概念-概念定义了relatedTopicOf关系。

\section{抽取结果与统计}
本节给出我们构建的跨语言知识库的知识数量统计结果。

%\vspace{-0.5cm}
\begin{table}[hb]
    \small
    \centering
    \caption{Statistics of Elementary Extraction Result}
    \label{tab:extract-result}
        \begin{tabular}{|p{2.5cm}|p{2cm}|p{2cm}|p{2cm}|p{2cm}|}
            \hline
                       & 英文维基    & 中文维基   & 互动百科    & 百度百科     \\ \hline
            \#Class    & 982,432   & 159,705  & 31,802    & 1300      \\ \hline
            \#Instance & 4,304,113 & 662,650  & 5,590,751 & 5,622,404 \\ \hline
            \#Property & 43,976    & 18,842   & 1187      & 139,634   \\ \hline
        \end{tabular}
\end{table}
%\vspace{-0.6cm}

%\vspace{-0.5cm}
\begin{table}[ht]
\small
\centering
\caption{Statistics of XLORE}
\label{tab:kb-result}
\begin{tabular}{|p{2cm}|p{1.5cm}|p{1.5cm}|p{1.5cm}|p{1.5cm}|p{1.5cm}|p{1.5cm}|}
\hline
\multicolumn{1}{|c|}{} & \multicolumn{2}{c|}{概念}     & \multicolumn{2}{c|}{实例}                   & \multicolumn{2}{c|}{属性}    \\ \hline
英文            & 639,020 & 96.26\%                & 3,879,121              & 38.79\%                & 15,380  & 27.24\%                \\ \hline
中文            & 88,615  & 13.35\%                & 7,409,519              & 68.25\%                & 51,618  & 91.44\%                \\ \hline
跨语言          & 63,895  & 9.63\%                 & 432,598                & 3.98\%                 & 10,549  & 18.69\%                \\ \hline
%English O       & 575,125 & 86.65\%                & 3,446,523              & 31.75\%                & 4,831   & 8.56\%    \\ \hline
%Chinese O       & 24,720  & 3.72\%                 & 6,976,921              & 64,27\%                & 41,069  & 72.75\%   \\ \hline
总数           & 663,740 & \multicolumn{1}{c|}{-} & 10,856,042             & \multicolumn{1}{c|}{-} & 56,449  & \multicolumn{1}{c|}{-} \\ \hline
\end{tabular}
\end{table}

Ttl生成

标准的owl与自定义xxx

\section{本章小结}
本章节介绍一个从多个在线百科中抽取数据,构成跨语言知识库的方法。我们从网页中抽取结构化数据并统一数据格式,生成并扩展跨语言链接信息来融合双语知识。为了精炼数据,我们还对语义信息进行判断与剪枝。最终获得的知识库XLore,包含663,740个概念,56,449个属性以及10,856,042个实例。

