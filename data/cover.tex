\thusetup{
  %******************************
  % 注意:
  %   1. 配置里面不要出现空行
  %   2. 不需要的配置信息可以删除
  %******************************
  %
  %=====
  % 秘级
  %=====
  secretlevel={秘密},
  secretyear={10},
  %
  %=========
  % 中文信息
  %=========
  ctitle={跨语言知识图谱中的概念属性生成与对齐},
  cdegree={工程硕士},
  cdepartment={计算机科学与技术系},
  cmajor={计算机技术},
  cauthor={李明洋},
  csupervisor={李涓子教授},
  %cassosupervisor={陈文光教授}, % 副指导老师
  %ccosupervisor={某某某教授}, % 联合指导老师
  % 日期自动使用当前时间,若需指定按如下方式修改:
   cdate={二〇一六年四月},
  %
  % 博士后专有部分
  cfirstdiscipline={计算机科学与技术},
  cseconddiscipline={系统结构},
  postdoctordate={2009年7月——2011年7月},
  id={编号}, % 可以留空: id={},
  udc={UDC}, % 可以留空
  catalognumber={分类号}, % 可以留空
  %
  %=========
  % 英文信息
  %=========
  etitle={Domain Property Generation and Alignment in Cross-lingual Knowledge Base},
  % 这块比较复杂,需要分情况讨论:
  % 1. 学术型硕士
  %    edegree:必须为Master of Arts或Master of Science(注意大小写)
  %             “哲学、文学、历史学、法学、教育学、艺术学门类,公共管理学科
  %              填写Master of Arts,其它填写Master of Science”
  %    emajor:“获得一级学科授权的学科填写一级学科名称,其它填写二级学科名称”
  % 2. 专业型硕士
  %    edegree:“填写专业学位英文名称全称”
  %    emajor:“工程硕士填写工程领域,其它专业学位不填写此项”
  % 3. 学术型博士
  %    edegree:Doctor of Philosophy(注意大小写)
  %    emajor:“获得一级学科授权的学科填写一级学科名称,其它填写二级学科名称”
  % 4. 专业型博士
  %    edegree:“填写专业学位英文名称全称”
  %    emajor:不填写此项
  edegree={Master of Engineering},
  emajor={Computer Technology},
  eauthor={Li Mingyang},
  esupervisor={Professor Li Juanzi},
  %eassosupervisor={Chen Wenguang},
  % 日期自动生成,若需指定按如下方式修改:
  edate={April, 2016},
  %
  % 关键词用“英文逗号”分割
  ckeywords={属性模板, 属性对齐, 知识图谱, 实体链接},
  ekeywords={Template, Property Alignment, Property Matching, Knowledge Base, Entity Linking}
}

% 定义中英文摘要和关键字
\begin{cabstract}
随着语义网的快速发展,知识图谱的研究也如火如荼。DBpedia,Yago等大型跨领域知识库的构建,从收集更多知识,向多语言特性演变。跨语言知识图谱逐渐受到人们的重视,其对全球知识分享起着重要作用。属性作为知识图谱中起着描述实体、关联实体的作用,它与概念一起,在本体中扮演着重要角色。给定一个概念,其领域下则有一系列属性来描述此概念下的实例的特性。

本文针对概念限定下的属性,即概念属性,展开研究,首先通过分析维基百科的信息框,生成一系列概念属性集合,之后在异构百科下进行概念属性跨语言对齐,最后将结果应用于跨语言知识图谱XLore上。

维基百科在信息框的组织中使用模板进行规范,该模板集合某一概念下的通用属性。通过维基信息框模板设计进行详尽的分析,对模板归类并制定抽取方法,最终生成大量概念属性信息,可用来描述90\%以上的维基词条,为属性的进一步研究提供了基础数据。

为解决中英文知识数量的差异问题,引入大型中文百科属性信息,提出异构百科下的跨语言属性对齐框架。该框架首先提出概念属性生成方法,从概念上对齐两百科信息;再次通过同语言对齐与同义词合并,解决百科差异性;然后通过各百科互对齐关系,首先获得一批高质量的跨语言对齐种子;最后提取文本、语义、分布上的特征,训练二分类器,在概念下进行跨语言链接的判断,发现更多新链接。

基于异构百科构建的跨语言知识库XLore,并将跨语言属性对齐结果应用在其中。为了直观地观察,在XLore知识库之上搭建展示系统XLore.org。该系统将知识进行了多角度多语言的展示与查询,包括页面展示、关系可视化、面向模糊搜索的知识查询与面向专业人士的SPARQL查询等。

在知识库的应用方面,提供了基于XLore的实体链接接口。该接口建立在一个多百科知识构建的文本—实体候选集上,支持中英文学术词汇、通用实体链接,并提供对应需求的实体消歧算法,如领域消歧、文本相似等。该接口最终应用在学术网站与新闻研究领域。

\end{cabstract}

% 如果习惯关键字跟在摘要文字后面,可以用直接命令来设置,如下:
% ckeywords={属性模板, 属性对齐, 知识图谱, 实体链接},

\begin{eabstract}
\end{eabstract}

% \ekeywords{\TeX, \LaTeX, CJK, template, thesis}
%ekeywords={Template, Property Alignment, Property Matching, Knowledge Base, Entity Linking}
